\documentclass[a4paper,12pt]{article}
\usepackage{amsmath}
\usepackage{amsfonts}
\usepackage{amssymb}
\usepackage{amsthm}
\usepackage{mathtools}
\usepackage[utf8]{inputenc}
\usepackage[a4paper, margin=2cm]{geometry}
\usepackage{hyperref}
\usepackage{calrsfs}
\usepackage{graphicx}
\usepackage{grffile}
\usepackage{float}
\usepackage{bm}
\usepackage{cite}
\usepackage{nicefrac}

\theoremstyle{definition}
\newtheorem{question}{Question}[section]
\newtheorem{definition}{Definition}[section]

\hypersetup{
	colorlinks=true,
	linkcolor=blue,
	filecolor=magenta,
	urlcolor=cyan,
}

\numberwithin{equation}{section}

\setlength{\parindent}{0em}
\setlength{\parskip}{1em}
\renewcommand{\baselinestretch}{1.15}

\allowbreak

% Subjective Logic macros

\usepackage{calrsfs}

% Domain
\newcommand{\dom}[1]{\mathbb{#1}}

% Powerset
\newcommand{\powset}[1]{\mathcal{P}(\mathbb{#1})}

% Hyperdomain
\newcommand{\hdom}[1]{\mathcal{R}(\mathbb{#1})}

% Composite set
\newcommand{\compset}[1]{\mathcal{C}(\mathbb{#1})}

% Base rate distribution
\newcommand{\ad}[1]{\mathbf{a}_{#1}}
\newcommand{\ada}[2]{\mathbf{a}^{#1}_{#2}}
\newcommand{\adx}[2]{\mathbf{a}_{#1}(#2)}
\newcommand{\adax}[3]{\mathbf{a}^{#1}_{#2}(#3)}

% Belief mass distribution
\newcommand{\bmd}[1]{\mathbf{b}_{#1}}
\newcommand{\bmda}[2]{\mathbf{b}^{#1}_{#2}}
\newcommand{\bmdx}[2]{\mathbf{b}_{#1}(#2)}
\newcommand{\bmdax}[3]{\mathbf{b}^{#1}_{#2}(#3)}

% Uncertainty mass
\newcommand{\ux}[1]{u_{#1}}
\newcommand{\uax}[2]{u^{#1}_{#2}}

% Projected Probability
\newcommand{\ppa}[2]{\mathbf{P}^{#1}_{#2}}

% Opinion
\newcommand{\opi}[1]{\omega_{#1}}
\newcommand{\opia}[2]{\omega^{#1}_{#2}}

\newcommand{\qm}[1]{`#1'}

\usepackage{xcolor}
\definecolor{darkgreen}{rgb}{0.0, 0.2, 0.13}
\definecolor{forestgreen(web)}{rgb}{0.13, 0.55, 0.13}
\definecolor{green(html/cssgreen)}{rgb}{0.0, 0.5, 0.0}

\newcommand{\red}{\textcolor{red}{red}}
\newcommand{\green}{\textcolor{green(html/cssgreen)}{green}}
\newcommand{\blue}{\textcolor{blue}{blue}}

%opening
\title{Subjective Logic}
\author{José C. Oliveira}

\begin{document}

\maketitle

\section{Introduction}

One of the major goals of our research is to develop a quantitative logic for reasoning about belief in social networks. Subjective logic is a logic that may has the expressiveness that we look for improve our influence graph and the belief state of an agent.

Subjective logic is an extension of probabilistic logic. In probabilistic logic, we can express the truth value of a proposition by a probability distribution over a domain with disjoint events or states and reason by the axioms of probability. When we have two states of a domain $\dom{X}$ and the probability of a state $x \in \dom{X}$ can be $P(x) = 0$ or $P(x) = 1$, we have binary logic. The expression $x \land y$ is expressed in probabilistic logic as $P(x \land y) = P(x)P(y)$. Probabilistic logic is an extension of binary logic.

Subjective logic extends probabilistic logic by adding \emph{uncertainty} and \emph{subjectivity}. We can't express \emph{\qm{we don't know}} with probabilistic logic by a uniform distribution because it says that we know that the distribution over the domain is uniform. Subjective logic can express that \emph{uncertainty} about the distribution. The \emph{subjectivity} comes from the fact that we can assign an opinion about a proposition to an agent.

By the investigation we are doing about subjective logic, our main question is: Can we use subjective logic to improve our model of social networks? If yes, how?

This document is an introduction about subjective logic and it will present elementary definitions, types of opinions and an overview about computational trust. The main reference of this document is the book \emph{Subjective Logic: A Formalism for Reasoning Under Uncertainty} by Audun Jøsang.


%\begin{itemize}
%	\item Probabilistic logic.
%	
%	\item Subjective logic as probabilistic logic with uncertainty and subjectivity.
%	
%	\item Main question of this investigation: Can we use Subjective Logic to improve the model?
%	
%	\item Summary.
%\end{itemize}

\section{Opinion representation}

The main object of subjective logic is the \emph{opinion}, and there are many equivalent ways for opinion representation. Here I'm going to present the most used one. We represent an opinion by $\opia{A}{X}$, where $A$ is the an agent and $X$ is a random variable or a \emph{hypervariable}. This section presents the elementary definitions that compose an opinion.

\subsection{Elementary definitions}

\subsubsection{Domain and hyperdomain}

In subjective logic, a domain is a state space consisting of a set of values which can also be called states, events, outcomes, hypotheses or propositions. Those values are assumed to be exclusive and exhaustive. Let $k = |\dom{X}|$ be the cardinality of $\dom{X}$.

Suppose we have a box have balls that can \red, \green, or \blue. Then, the domain that represents all possible outcomes is
\begin{equation}
	\dom{X} = \{\red, \green, \blue\}\text{.}
\end{equation}


\begin{definition}
	 \emph{(Hyperdomain)} Let $\dom{X}$ be a domain, and let $\powset{X}$ denote the powerset of $\dom{X}$. The powerset contains all subsets of $\dom{X}$, including the empty set $\emptyset$, and the domain $\dom{X}$ itself. The \emph{hyperdomain} denoted $\hdom{X}$ is the reduced powerset of $\dom{X}$, i.e. the powerset excluding the empty-set $\varnothing$ and the domain value $\dom{X}$. The hyperdomain is expressed as
	\begin{equation}
		\text{Hyperdomain:}\ \mathcal{R}(\mathbb{X}) = \mathcal{P} \setminus \{\mathbb{X}, \emptyset\}
	\end{equation}
\end{definition}

The hyperdomain of the box is
\begin{equation}
    \begin{array}{rll}
        \hdom{\dom{X}} = \{ & \{\red\}, \{\green\}, \{\blue\}, \\
        & \{\red, \green\}, \{\red, \blue\}, \{\green, \blue\} & \}\text{.}
    \end{array}
\end{equation}

Let $\kappa = |\hdom{X}| = 2^k -2\text{.}$ be the cardinality of $\hdom{R}$.

Every value of the hyperdomain with one value is called \emph{singleton}. Every value with more the one value is called \emph{composite value}. The interpretation of a composite value being TRUE, is that one and only one of the constituent singletons is TRUE. The set of all composite values $\compset{X}$ is called \emph{composite set}.

A \emph{hypervariable} $X$ is an random variable that its values from $\hdom{X}$. For example, if a hypervariable takes value from the composite value $\{\blue, \red\} \in \hdom{X}$, it means that we either draw a ball \blue\ or \red.

\subsubsection{Belief mass distribution and uncertainty mass}

\emph{Belief mass} can be distributed over a (hyper)domain. Assign belief mass to a singleton value $x \in \dom{X}$ expresses support for $x$ being TRUE. Assign belief mass to a composite value $x \in \compset{X}$ express support for one of the singleton values contained in $x$ being TRUE. Belief mass is sub-additive and is complemented by \emph{uncertainty mass} $\ux{X}$ and it represents
vacuity of evidence, i.e. the lack of support for the variable $X$ to have any specific value.

\begin{definition}
	\emph{(Belief Mass Distribution)} Let $\mathbb{X}$ be a domain with corresponding hyperdomain $\mathcal{R}(\mathbb{X})$, and let $X$ be a variable over those domains. A belief mass distribution denote $\mathbf{b}_X$ assigns belief mass to possible values of the variable $X$. In the case of a random variable $X \in \mathbb{X}$, the belief mass distribution applies to domain $\mathbb{X}$, and in the case of a hypervariable $X \in \mathcal{R}(\mathbb{X})$ the belief mass distribution applies to hyperdomain $\mathcal{R}(\mathbb{X})$. This is formally defined as follows.
	\begin{equation}\label{eq:multinomial-belief-mass-dristribution}
		\begin{matrix*}[l]
			\text{Multinomial belief mass distribution:}\ \mathbf{b}_X : \mathbb{X} \rightarrow [0,\ 1], \\
			\text{with the additivity requirement:}\ u_X + \sum\limits_{x \in \mathbb{X}} \mathbf{b}_X(x) = 1\text{.}
		\end{matrix*}
	\end{equation}
	\begin{equation}\label{eq:hypernomal_belief_mass_distribution}
		\begin{matrix*}[l]
			\text{Hypernominal belief mass distribution:}\ \mathbf{b}_X : \mathcal{R}(\mathbb{X}) \rightarrow [0,\ 1], \\
			\text{with the additivity requirement:}\ u_X + \sum\limits_{x \in \mathcal{R}(\mathbb{X})} \mathbf{b}_X(x) = 1\text{.}
		\end{matrix*}
	\end{equation}
\end{definition}

\subsubsection{Base rate distribution}

Base rate distribution represents a \emph{prior} probability distribution over a domain, i.e., when we are not considering any evidence and we are completely uncertain.

\begin{definition}\label{def:base_rate_distribution}
	\emph{(Base Rate Distribution)} Let $\mathbb{X}$ be a domain, and let $X$ be a random variable in $\mathbb{X}$. The base rate distribution $\mathbf{a}_X$ assigns base rate probability to possible values of $X \in \mathbb{X}$, and is an additive probability distribution, formally expressed as:
	\begin{equation}\label{eq:base_rate_distribution}
		\begin{matrix*}[l]
			\text{Base rate distribution:}\ \mathbf{a}_X : \mathbb{X} \rightarrow [0,\ 1], \\
			\text{with the additivity requirement:}\ \sum\limits_{x \in \mathbb{X}} \mathbf{a}_X(x) = 1\text{.}
		\end{matrix*}
	\end{equation}
\end{definition}

\begin{definition}\label{def:base_rate_distribution_over_values_in_a_hyperdomain}
	\emph{(Base Rate Distribution over Values in a Hyperdomain)} Let $\mathbb{X}$ be a domain with corresponding hyperdomain $\mathcal{R}(\mathbb{X})$, and let $X$ be a variable over those domains. Assume the base rate distribution $\mathbf{a}_X$ over the domain $\mathbb{X}$ according to Definition~\ref{def:base_rate_distribution}. The base rate $\mathbf{a}_X$ for a composite value $x \in \mathcal{R}(\mathbb{X})$ can be computed as follows:
	\begin{equation}
		\text{Base rate over composite values:}\ \mathbf{a}_X(x_i) = \sum\limits_{\substack{x_j \in \mathbb{X} \\ x_j \subseteq x_i}} \mathbf{a}_X(x_j),\ \forall x_i \in \mathcal{R}(\mathbb{X})\text{.}
	\end{equation}
\end{definition}

\begin{definition}
	\emph{(Relative Base Rate)} Assume a domain $\mathbb{X}$ of cardinality $k$, and the corresponding hyperdomain $\mathcal{R}(\mathbb{X})$. Let $X$ be a hypervariable over $\mathcal{R}(\mathbb{X})$. Assume that a base rate distribution $\mathbf{a}_X$ is defined over $\mathbb{X}$ according to Definition~\ref{def:base_rate_distribution_over_values_in_a_hyperdomain}. Then the base rate of a value $x$ relative to a value $v_i$ is expressed as the relative base rate $\mathbf{a}_X(x|x_i)$ defined below.
	\begin{equation}
		\mathbf{a}_X(x|x_i) = \dfrac{\mathbf{a}_X(x \cap x_i)}{\mathbf{a}_X(x_i)}\text{, } \forall x, x_i \in \mathcal{R}(\mathbb{X}) \text{, where}\ \mathbf{a}_X(x_i) \neq 0\text{.}
	\end{equation}

	In the case when $\mathbf{a}_X(x_i) = 0$, then $\mathbf{a}_X(x|x_i) = 0$. Alternatively it can simply be assumed that $a_X(x_i) > 0$, for every $x_i \in \mathbb{X}$, meaning that everything we include in the domain has a non-zero base rate of occurrence in general.
\end{definition}

Defining base rate distribution over a domain depends of the background of the given situation. For our box, the base rate distribution is uniform. Now suppose that we have a disease that happens in $10\%$ of the population. Let $\dom{Y} = \{y, \overline{y}\}$ be the domain representing the occurrence or not of the disease on a person. Let $Y$ be a random variable over $\dom{Y}$. The base rate distribution for a untested person for the disease is $\ad{Y} = (0.05, 0.95)$.

\subsubsection{Projected probability}

The \emph{projected probability} is a \emph{posterior} probability distribution over the domain that considers the base rate distribution (\emph{prior}), the belief mass distribution (evidence) and the uncertainty mass (lack of evidence). The projected probability distribution is different for every type of opinion, and it is explained below.

\subsection{Binomial opinions}

A binary opinion states about a domain with $k = 2$, and any multinomial opinion, with $k > 2$, can be considered binary when seen as a binary partition consisting of a proper subset $x \subset \dom{X}$ and its complement $\overline{x}$.

\begin{definition}
	\emph{Binomial Opinion} Let $\mathbb{X} = \{x, \overline{x}\}$ be a binary domain with binomial random variable $X \in \mathbb{X}$. A binomial opinion about the truth/presence of value $x$ is the ordered quadruplet $\omega_x = \left(b_x, d_x, u_x, a_x\right)$, where the additivity requirement
	\begin{equation}
		b_x + d_x + u_x = 1
	\end{equation}
	is satisfied, and where the respective parameters are defined as
	\begin{itemize}
		\item $b_x$: \emph{belief mass} in support of $x$ being TRUE (i.e. $X = x$),
		\item $d_x$: \emph{disbelief mass} in support of $x$ being FALSE (i.e. $X = \overline{x}$)
		\item $u_x$: \emph{uncertainty mass} representing the vacuity of evidence,
		\item $a_x$: \emph{base rate}, i.e. prior probability of $x$ without any evidence.
	\end{itemize}
\end{definition}

%Let $\dom{Y} = \{y, \overline{y}\}$ be the domain representing the occurrence or not of the disease on a person, like our previous example. Let $Y$ be a random variable over $\dom{Y}$. Let the base rate for a untested person have the disease be $0.9$. Now, suppose the a test indicated

The projected probability distribution of binomial opinions is defined by:
\begin{equation}
	P(x) = b_x + a_x u_x
\end{equation}

Note that the more the opinion relies on uncertainty, the more weight the base rate will have on the projected probability.

Let $\dom{Z} = \{\red, \green\}$ be a domain the represents the outcomes of drawing a ball from a box. Let $Z$ be a random variable over $\dom{Y}$. Let the base rate be uniform. Then, the opinion for drawing a \red\ ball (and \green\ ball too) is:
\begin{equation}
	\opi{Z} = (0, 0, 1, 0.5)
\end{equation}

Now, I told you that there is 10 balls in the box, 4 of then is \red\ and non is know about the other 6. The opinion for drawing a \red\ ball is:
\begin{equation}
	\opi{Z} = (0.4, 0, 0.6, 0.5)
\end{equation}

The projected probability distribution is:
\begin{equation}
	\begin{array}{lll}
		P(\red) & = 0.4 + 0.5 \cdot 0.6 & = 0.7 \\
		P(\green) & = 0 + 0.6 \cdot 0.5 & = 0.3
	\end{array}
\end{equation}

\subsection{Multinomial opinions}

Multinomial opinions represent the generalisation of binomial opinions. Multinomial opinions apply to situations where a random variable $X$ over $\dom{X}$ can take one of multiple different values.

\begin{definition}
	\emph{(Multinomial Opinion)} Let $\mathbb{X}$ be a domain larger than binary, i.e. so that $k = |X| > 2$. Let $X$ be a random variable in $\mathbb{X}$. A multinomial opinion over the random variable $X$ is the ordered triplet $\omega_X = (\mathbf{b}_X, u_X , \mathbf{a}_X)$ where
	\begin{itemize}
		\item $\mathbf{b}_X$ is a belief mass distribution over $X$,
		\item $u_X$ is the uncertainty mass which represents the vacuity of evidence,
		\item $\mathbf{a}_X$ is a base rate distribution over $\mathbb{X}$,
	\end{itemize}
	and the multinomial additivity requirement of Eq.(\ref{eq:multinomial-belief-mass-dristribution}) is satisfied.
\end{definition}

The projected probability distribution of multinomial opinions is defined by:
\begin{equation}\label{eq:multinomial_projected_probability}
	\mathbf{P}_X(x) = \mathbf{b}_X(x) + \mathbf{a}_X(x) u_X,\ \forall x \in \mathbb{X}\text{.}
\end{equation}

Let $\dom{X} = \{\red, \green, \blue\}$ be the domain representing the outcomes of drawing a ball from a box. Let $X$ be a random variable over $\dom{X}$. Let the base rate distribution be uniform. Now, suppose that we know that there is $3$ \red\ balls, $2$ \blue\ balls and we know nothing about the other 5 balls. Then, the opinion is:
\begin{equation}
	\opi{X} =
	\left(\begin{array}{lllll}
		\bmdx{X}{\red} & = 0.3, & \ & \adx{X}{\red} & = \nicefrac{1}{3}, \\
		\bmdx{X}{\green} & = 0, & \ & \adx{X}{\green} & = \nicefrac{1}{3}, \\
		\bmdx{X}{\blue} & = 0.2, & \ & \adx{X}{\blue} & = \nicefrac{1}{3}, \\
		\ux{X} & = 0.5.
	\end{array}\right)
\end{equation}

The projected probability distribution is:
\begin{equation}
	\begin{array}{lll}
		\mathbf{P}(\red) & =  0.3 + \nicefrac{1}{3} \cdot 0.5 & \approx 0.466 \\
		\mathbf{P}(\green) & = 0 + \nicefrac{1}{3} \cdot 0.5 & \approx 0.166 \\
		\mathbf{P}(\blue) & = 0.2 + \nicefrac{1}{3} \cdot 0.5 & \approx 0.366  
	\end{array}
\end{equation}


\subsection{Hypernomial opinions}

Hyper-opinions represents the generalisation of multinomial opinions. In the case of a domain $\dom{X}$ with hyperdomain $\hdom{X}$, it is possible to obtain evidence for a composite value $x \in \hdom{X}$, which translates into assigning belief mass to that composite value.

\begin{definition}
    \emph{(Hyper-opinion)} Let $\mathbb{X}$ be a domain of cardinality $k > 2$, with corresponding hyperdomain $\mathcal{R}(\mathbb{X})$. Let $X$ be a hypervariable in $\mathcal{R}(\mathbb{X})$. A hyper-opinion on the hypervariable $X$ is the ordered triplet $\omega_X =(\mathbf{b}_X, u_X , \mathbf{a}_X)$ where
    \begin{itemize}
        \item $\mathbf{b}_X$ is a belief mass distribution over $\mathcal{R}(\mathbb{X})$,
        \item $u_X$ is the uncertainty mass which represents the vacuity of evidence,
        \item $\mathbf{a}_X$ is a base rate distribution over $\mathbb{X}$,
    \end{itemize}
    and the hypernomial additivity of Eq.(\ref{eq:hypernomal_belief_mass_distribution}) is satisfied.
\end{definition}

The projected probability distribution of hyper-opinions is defined by:
\begin{equation}
    \mathbf{P}_X(x) = \sum\limits_{x_i \in \mathcal{R}(\mathbb{X})} \mathbf{a}_X(x|x_i) \mathbf{b}_X(x_i) + \mathbf{a}_X(x) u_X \text{, } \forall x \in \mathbb{X}.
\end{equation}

Let $\dom{X} = \{\red, \green, \blue\}$ be the domain representing the outcomes of drawing a ball from a box. The hyperdomain is:
\begin{equation}
    \begin{array}{rll}
        \hdom{\dom{X}} = \{ & \{\red\}, \{\green\}, \{\blue\}, \\
        & \{\red, \green\}, \{\red, \blue\}, \{\green, \blue\} & \}\text{.}
    \end{array}
\end{equation}

Let $X$ be a hypervariable over $\hdom{X}$. Let the base rate distribution be uniform. Now, suppose that we know that the box have one \red\ ball, $2$ \blue\ balls, and $3$ balls \green\ or \blue. We know nothing about $4$ balls. Then, the hyperopinion is:
\begin{equation}
    \opi{X} = \left(
        \begin{array}{lllll}
            \bmdx{X}{\{\red\}} & = 0.1, & \ & \adx{X}{\{\red\}} & = \nicefrac{1}{3}, \\
            \bmdx{X}{\{\green\}} & = 0, & \ & \adx{X}{\{\green\}} & = \nicefrac{1}{3}, \\
            \bmdx{X}{\{\blue\}} & = 0.2, & \ & \adx{X}{\{\blue\}} & = \nicefrac{1}{3}, \\
            \bmdx{X}{\{\red, \green\}} & = 0, & \ & \adx{X}{\{\red, \green\}} & = \nicefrac{2}{3}, \\
            \bmdx{X}{\{\red, \blue\}} & = 0, & \ & \adx{X}{\{\red, \blue\}} & = \nicefrac{2}{3}, \\
            \bmdx{X}{\{\green, \blue\}} & = 0.3, & \ & \adx{X}{\{\green, \blue\}} & = \nicefrac{2}{3}, \\
            \ux{X} & = 0.4. & &
        \end{array}
    \right)
\end{equation}

The projected probability distribution is:
\begin{equation}
	\begin{array}{llllll}
		\mathbf{P}(\{\red\}) & \approx 0.2333 & \ & \mathbf{P}(\{\red, \green\}) & \approx 0.5166\\
		\mathbf{P}(\{\green\}) & \approx 0.2833 & \ & \mathbf{P}(\{\red, \blue\}) & \approx 0.7166 \\
		\mathbf{P}(\{\blue\}) & \approx 0.4833 & \ & \mathbf{P}(\{\green, \blue\}) & \approx 0.7666
	\end{array}
\end{equation}

%\begin{equation}
%	\begin{array}{lll}
%		P(\{\red\}) & =  (1 \cdot 0.1) + \nicefrac{1}{3} \cdot 0.4 & \approx 0.2333 \\
%		P(\{\green\}) & = () + \nicefrac{1}{3} \cdot 0.4 & \approx 0.2833 \\
%		P(\{\blue\}) & = 0.2 + \nicefrac{1}{3} \cdot 0.5 & \approx 0.366  
%	\end{array}
%\end{equation}







%\begin{itemize}
%	\item Elementary definitions
%	
%	\begin{itemize}
%		\item Domain and Hyperdomain
%		
%		\item Base-rate distribution (prior)
%		
%		\item Belief mass distribution and uncertainty
%		
%		\item Projected probability distribution (posterior)
%	\end{itemize}
%	
%	\item Binomial opinion and example
%	
%	\item Multinomial option and example
%	
%	\item Hypernomial opinion and example
%	
%\end{itemize}

\section{Computational trust}

This is an overview. Nothing formal. (Maybe it will be if I have time.)

\begin{itemize}
	\item Definition of trust. (Influence?)
	
	\item Trust transitivity. (Update function?)
	
	\item Belief fusion. (Overall update function?)
\end{itemize}

\section{Next questions}

When I was reviewing the book, I found something that I may have got wrong about trust transitivity. I'm not sure if the example I told at the meeting is possible in SL.

\begin{itemize}
	\item Suppose that an agent A trusts an agent B ($\omega^A_B$), and B trusts X ($\omega^A_X$). Can the carnality of the domain of $\omega^A_X$ be greater than 2? (At the meeting, assumed yes. Now I'm not sure.)
	
	\item Is there a way (an operator) to consider the trust of an agent A when they obtain another trust opinion (from trust transitivity)?
	
	\begin{itemize}
		\item If yes, can this operator have the same properties as the rational belief update?
		
		\item If not, can we create an operator that has the same properties as the rational belief update?
	\end{itemize}
\end{itemize}


%\bibliographystyle{plain}
%\bibliography{bibliography.bib}


\end{document}