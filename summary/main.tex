\documentclass[a4paper,12pt]{article}
\usepackage{amsmath}
\usepackage{amsfonts}
\usepackage{amssymb}
\usepackage{amsthm}
\usepackage{mathtools}
\usepackage[utf8]{inputenc}
\usepackage[a4paper, margin=2cm]{geometry}
\usepackage{hyperref}
\usepackage{calrsfs}
\usepackage{graphicx}
\usepackage{grffile}
\usepackage{float}
\usepackage{bm}
\usepackage{cite}

\theoremstyle{definition}
\newtheorem{question}{Question}[section]
\newtheorem{definition}{Definition}[section]

\hypersetup{
	colorlinks=true,
	linkcolor=blue,
	filecolor=magenta,
	urlcolor=cyan,
}

\numberwithin{equation}{section}

\setlength{\parindent}{0em}
\setlength{\parskip}{1em}
\renewcommand{\baselinestretch}{1.15}

\allowbreak

% Subjective Logic macros

\usepackage{calrsfs}

% Domain
\newcommand{\dom}[1]{\mathbb{#1}}

% Powerset
\newcommand{\powset}[1]{\mathcal{P}(\mathbb{#1})}

% Hyperdomain
\newcommand{\hdom}[1]{\mathcal{R}(\mathbb{#1})}

% Composite set
\newcommand{\compset}[1]{\mathcal{C}(\mathbb{#1})}

% Base rate distribution
\newcommand{\ad}[1]{\mathbf{a}_{#1}}
\newcommand{\ada}[2]{\mathbf{a}^{#1}_{#2}}
\newcommand{\adx}[2]{\mathbf{a}_{#1}(#2)}
\newcommand{\adax}[3]{\mathbf{a}^{#1}_{#2}(#3)}

% Belief mass distribution
\newcommand{\bmd}[1]{\mathbf{b}_{#1}}
\newcommand{\bmda}[2]{\mathbf{b}^{#1}_{#2}}
\newcommand{\bmdx}[2]{\mathbf{b}_{#1}(#2)}
\newcommand{\bmdax}[3]{\mathbf{b}^{#1}_{#2}(#3)}

% Uncertainty mass
\newcommand{\ux}[1]{u_{#1}}
\newcommand{\uax}[2]{u^{#1}_{#2}}

% Projected Probability
\newcommand{\ppa}[2]{\mathbf{P}^{#1}_{#2}}

% Opinion
\newcommand{\opi}[1]{\omega_{#1}}
\newcommand{\opia}[2]{\omega^{#1}_{#2}}

\newcommand{\qm}[1]{``#1''}

\usepackage{xcolor}
\definecolor{darkgreen}{rgb}{0.0, 0.2, 0.13}
\definecolor{forestgreen(web)}{rgb}{0.13, 0.55, 0.13}
\definecolor{green(html/cssgreen)}{rgb}{0.0, 0.5, 0.0}

\newcommand{\red}{\textcolor{red}{red}}
\newcommand{\green}{\textcolor{green(html/cssgreen)}{green}}
\newcommand{\blue}{\textcolor{blue}{blue}}

%opening
\title{Subjective Logic}
\author{José C. Oliveira}

\begin{document}

\maketitle

\section{Introduction}

One of the major goals of our research is to develop a quantitative logic for reasoning about belief in social networks. Subjective logic is a logic that may has the expressiveness that we look for improve our influence graph and the belief state of an agent.

Subjective logic is an extension of probabilistic logic. In probabilistic logic, we can express the truth value of a proposition by a probability distribution over a domain with disjoint events or states and reason by the axioms of probability. When we have two states of a domain $\dom{X}$ and the probability of a state $x \in \dom{X}$ can be $P(x) = 0$ or $P(x) = 1$, we have binary logic. The expression $x \land y$ is expressed in probabilistic logic as $P(x \land y) = P(x)P(y)$. Probabilistic logic is an extension of binary logic.

Subjective logic extends probabilistic logic by adding \emph{uncertainty} and \emph{subjectivity}. We can't express \emph{\qm{we don't know}} with probabilistic logic by a uniform distribution because it says that we know that the distribution over the domain is uniform. Subjective logic can express that \emph{uncertainty} about the distribution. The \emph{subjectivity} comes from the fact that we can assign an opinion about a proposition to an agent.

By the investigation that we are doing about subjective logic, our main question is: Can we use subjective logic to improve our model of social networks? If yes, how?

This document is an introduction about subjective logic and it will present elementary definition, types of opinions and an overview about computational trust. The main reference of this documents if the book \emph{Subjective Logic: A Formalism for Reasoning Under Uncertainty} by Audun Jøsang.


%\begin{itemize}
%	\item Probabilistic logic.
%	
%	\item Subjective logic as probabilistic logic with uncertainty and subjectivity.
%	
%	\item Main question of this investigation: Can we use Subjective Logic to improve the model?
%	
%	\item Summary.
%\end{itemize}

\section{Opinion representation}

The main object of subjective logic is the \emph{opinion}. There are many equivalent ways for opinion representation. Here I'm going to present the most used. We represent a opinion by $\opia{A}{X}$, where $A$ is the an agent e $X$ is a random variable or \emph{hypervariable}. This section presents the elementary definitions that form an opinion.

\subsection{Elementary definitions}

In subjective logic, a domain is a state space consisting of a set of values which can also be called states, events, outcomes, hypotheses or propositions. Those values are assumed to be exclusive and exhaustive. Let $k = |\dom{X}|$.

Suppose a box have balls that can \red, \green, or \blue. Then, the domain is
\begin{equation}
	\dom{X} = \{\red, \green, \blue\}\text{.}
\end{equation}


\begin{definition}
	 \emph{(Hyperdomain)} Let $\dom{X}$ be a domain, and let $\powset{X}$ denote the powerset of $\dom{X}$. The powerset contains all subsets of $\dom{X}$, including the empty set $\emptyset$, and the domain $\dom{X}$ itself. The \emph{hyperdomain} denoted $\hdom{X}$ is the reduced powerset of $\dom{X}$, i.e. the powerset excluding the empty-set $\varnothing$ and the domain value $\dom{X}$. The hyperdomain is expressed as
	\begin{equation}
		\text{Hyperdomain:}\ \mathcal{R}(\mathbb{X}) = \mathcal{P} \setminus \{\mathbb{X}, \emptyset\}
	\end{equation}
\end{definition}

The hyperdomain of the box is
\begin{equation}
    \begin{array}{rll}
        \hdom{\dom{X}} = \{ & \{\red\}, \{\green\}, \{\blue\}, \\
        & \{\red, \green\}, \{\red, \blue\}, \{\green, \blue\} & \}\text{.}
    \end{array}
\end{equation}

Let $\kappa = |\hdom{X}| = 2^k -2\text{.}$

Every value of the hyperdomain with one value is called \emph{singleton}. Every value with more the one value is called \emph{composite value}. The interpretation of a composite value being TRUE, is that one and only one of the constituent singletons is TRUE. 

\subsection{Binomial opinions}

\subsection{Multinomial opinions}

\subsection{Hypernomial opinions}




\begin{itemize}
	\item Elementary definitions
	
	\begin{itemize}
		\item Domain and Hyperdomain
		
		\item Base-rate distribution (prior)
		
		\item Belief mass distribution and uncertainty
		
		\item Projected probability distribution (posterior)
	\end{itemize}
	
	\item Binomial opinion and example
	
	\item Multinomial option and example
	
	\item Hypernomial opinion and example
	
\end{itemize}

\section{Computational trust}

This is an overview. Nothing formal. (Maybe it will be if I have time.)

\begin{itemize}
	\item Definition of trust. (Influence?)
	
	\item Trust transitivity. (Update function?)
	
	\item Belief fusion. (Overall update function?)
\end{itemize}

\section{Next questions}

When I was reviewing the book, I found something that I may have got wrong about trust transitivity. I'm not sure if the example I told at the meeting is possible in SL.

\begin{itemize}
	\item Suppose that an agent A trusts an agent B ($\omega^A_B$), and B trusts X ($\omega^A_X$). Can the carnality of the domain of $\omega^A_X$ be greater than 2? (At the meeting, assumed yes. Now I'm not sure.)
	
	\item Is there a way (an operator) to consider the trust of an agent A when they obtain another trust opinion (from trust transitivity)?
	
	\begin{itemize}
		\item If yes, can this operator have the same properties as the rational belief update?
		
		\item If not, can we create an operator that has the same properties as the rational belief update?
	\end{itemize}
\end{itemize}


\bibliographystyle{plain}
\bibliography{bibliography.bib}


\end{document}