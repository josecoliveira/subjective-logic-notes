\documentclass[a4paper,12pt]{article}
\usepackage{amsmath}
\usepackage{amsfonts}
\usepackage{amssymb}
\usepackage{amsthm}
\usepackage{mathtools}
\usepackage[utf8]{inputenc}
\usepackage[a4paper, margin=2cm]{geometry}
\usepackage{hyperref}
\usepackage{calrsfs}
\usepackage{graphicx}
\usepackage{grffile}
\usepackage{float}
\usepackage{bm}

\theoremstyle{definition}
\newtheorem{question}{Question}[section]
\newtheorem{definition}{Definition}[section]

\hypersetup{
	colorlinks=true,
	linkcolor=blue,
	filecolor=magenta,
	urlcolor=cyan,
}

\numberwithin{equation}{section}

\setlength{\parindent}{0em}
\setlength{\parskip}{1em}
\renewcommand{\baselinestretch}{1.15}

\allowbreak

%opening
\title{Subjective Logic}
\author{José C. Oliveira}

\begin{document}

\maketitle

\section{Introduction}

\begin{itemize}
	\item Probabilistic logic.
	
	\item Subjective logic as probabilistic logic with uncertainty and subjectivity.
	
	\item Main question of this investigation: Can we use Subjective Logic to improve the model?
	
	\item Summary.
\end{itemize}

\section{Opinion representation}

\begin{itemize}
	\item Elementary definitions
	
	\begin{itemize}
		\item Domain and Hyperdomain
		
		\item Belief mass distribution (prior) and uncertainty
		
		\item Projected probability distribution (posterior)
	\end{itemize}
	
	\item Binomial opinion and example
	
	\item Multinomial option and example
	
	\item Hypernomial opinion and example
	
\end{itemize}

\section{Computational trust}

This is an overview. Nothing formal. (Maybe it will be if I have time.)

\begin{itemize}
	\item Definition of trust. (Influence)
	
	\item Trust transitivity. (Update function)
	
	\item Belief fusion. (Overall update function)
\end{itemize}

\section{Next questions}

When I was reviewing the book, I found something that I may have got wrong about trust transitivity. I'm not sure if the example I told at the meeting is possible in SL.

\begin{itemize}
	\item Suppose that an agent A trusts an agent B ($\omega^A_B$), and B trusts X ($\omega^A_X$). Can the carnality of the domain of $\omega^A_X$ be greater than 2? (At the meeting, assumed yes. Now I'm not sure.)
	
	\item Is there a way (an operator) to consider the trust of an agent A when they obtain another trust opinion (from trust transitivity)?
	
	\begin{itemize}
		\item If yes, can this operator have the same properties as the rational belief update?
		
		\item If not, can we create an operator that has the same properties as the rational belief update?
	\end{itemize}
\end{itemize}

\end{document}