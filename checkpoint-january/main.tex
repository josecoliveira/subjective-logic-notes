\documentclass[a4paper,12pt]{article}
\usepackage{amsmath}
\usepackage{amsfonts}
\usepackage{amssymb}
\usepackage{amsthm}
\usepackage{mathtools}
\usepackage[utf8]{inputenc}
\usepackage[a4paper, margin=2cm]{geometry}

\usepackage[backend=biber,citestyle=numeric,bibstyle=numeric]{biblatex}


\usepackage{hyperref}
\usepackage{calrsfs}
\usepackage{graphicx}
\usepackage{grffile}
\usepackage{float}
\usepackage{bm}
\usepackage{nicefrac}
\usepackage{chngcntr}

\counterwithin{figure}{section}

\theoremstyle{definition}
\newtheorem{question}{Question}[section]
\newtheorem{definition}{Definition}[section]

\theoremstyle{theorem}
\newtheorem{theorem}{Theorem}[section]

\hypersetup{
	colorlinks=true,
	linkcolor=blue,
	filecolor=magenta,
	urlcolor=cyan,
}

\numberwithin{equation}{section}

\setlength{\parindent}{0em}
\setlength{\parskip}{1em}
\renewcommand{\baselinestretch}{1.15}

\allowbreak

% Subjective Logic macros

\usepackage{calrsfs}

% Domain
\newcommand{\dom}[1]{\mathbb{#1}}

% Powerset
\newcommand{\powset}[1]{\mathcal{P}(\mathbb{#1})}

% Hyperdomain
\newcommand{\hdom}[1]{\mathcal{R}(\mathbb{#1})}

% Composite set
\newcommand{\compset}[1]{\mathcal{C}(\mathbb{#1})}

% Base rate distribution
\newcommand{\ad}[1]{\mathbf{a}_{#1}}
\newcommand{\ada}[2]{\mathbf{a}^{#1}_{#2}}
\newcommand{\adx}[2]{\mathbf{a}_{#1}(#2)}
\newcommand{\adax}[3]{\mathbf{a}^{#1}_{#2}(#3)}

% Belief mass distribution
\newcommand{\bmd}[1]{\mathbf{b}_{#1}}
\newcommand{\bmda}[2]{\mathbf{b}^{#1}_{#2}}
\newcommand{\bmdx}[2]{\mathbf{b}_{#1}(#2)}
\newcommand{\bmdax}[3]{\mathbf{b}^{#1}_{#2}(#3)}

% Uncertainty mass
\newcommand{\ux}[1]{u_{#1}}
\newcommand{\uax}[2]{u^{#1}_{#2}}

% Projected Probability
\newcommand{\ppa}[2]{\mathbf{P}^{#1}_{#2}}

% Opinion
\newcommand{\opi}[1]{\omega_{#1}}
\newcommand{\opia}[2]{\omega^{#1}_{#2}}

\newcommand{\qm}[1]{`#1'}

\usepackage{xcolor}
\definecolor{darkgreen}{rgb}{0.0, 0.2, 0.13}
\definecolor{forestgreen(web)}{rgb}{0.13, 0.55, 0.13}
\definecolor{green(html/cssgreen)}{rgb}{0.0, 0.5, 0.0}

\newcommand{\red}{\textcolor{red}{red}}
\newcommand{\green}{\textcolor{green(html/cssgreen)}{green}}
\newcommand{\blue}{\textcolor{blue}{blue}}

%opening
\title{A Model for Social Networks with Subjective Logic\\
\large{What we have until now}}
\author{José C. Oliveira}

\addbibresource{../bibliography.bib}

\begin{document}

\maketitle

\section{Introduction}

At \cite{DBLP:conf/birthday/AlvimKV19}, \citeauthor{DBLP:conf/birthday/AlvimKV19} developed a formal model for group polarization in social networks, and show possible future directions. Here, we are trying to develop a quantitative logic for reasoning about beliefs in social networks. With a logic, we can provide a formal way of making statements about knowledge.


\section{Subjective logic}

Subjective logic \cite{DBLP:books/sp/Josang16} is a extension of probabilistic logic expressed by opinions with \emph{uncertainty} about the probability distribution and \emph{subjectivity} by assigning an agent or source for the opinion. The following topics about subjective logic are important for this research.

\begin{itemize}
\item Elements of subjective logic
	\begin{itemize}
	\item Domains and hyperdomains
	\item Random variables and hypervariables
	\item Belief mass distribution and uncertainty mass
	\item Base rate distribution
	\item Probability distribution
	\end{itemize}
\item Opinion Representation
	\begin{itemize}
	\item Opinion classes
	\item Aleatory and epistemic opinions
	\item Binomial opinions
	\item Multinomial opinions
	\item Hyper-opinions
	\item Mapping between opinions and Dirichlet PDF
	\end{itemize}
\item Principles of subjective logic
	\begin{itemize}
	\item Comparison with related frameworks for uncertain reasoning
	\item Subjective logic as a generalization of probabilistic logic
	\end{itemize}
\item Belief fusion
	\begin{itemize}
	\item Interpretation and criteria for fusion operator selection
	\item Cumulative belief fusion
	\item Averaging belief fusion
	\item Weighted belief fusion
	\end{itemize}
\item Computational trust
	\begin{itemize}
	\item Notion of trust
	\item The trust-discounting operator
	\item Trust fusion
	\end{itemize}
\end{itemize}


\section{The model so far}

This is a very short summary. Let $\hdom{X}$ be a hyperdomain and $X$ be a hypervariable over $\hdom{X}$. Let $A_1, \cdots, A_n$ be agents. All agents have a hyperopinion $\opia{A_i}{X}$. Also, for every $i \in {1, \cdots, n}$ and for every $j \in {1, \cdots, n} \setminus {i}$, the agent $A_i$ has a trust opinion about $A_j$, $\opia{A_i}{A_j}$.

The idea is that an agent $A_i$ learns a new opinion by interaction with everyone else considering trust-discount and trust fusion. Then, we should merge $A_i$ previous opinion with the new learned opinion. Since there are several belief fusion operators, \textbf{we need to choose which suits best for our case}. Some of the question are the following:
\begin{itemize}
\item If an agent learns from two equal opinions, should the new opinion have more confidence? That is, should our belief fusion operator be idempotent or non-idempotent. If it should be non-idempotent, we choose cumulative belief fusion. It means that we are taking the sum of the amount of evidence from the two opinions. Otherwise, we move to the next question.
\item If an agent learns from two opinions with uncertainty, should the uncertainty of the new opinion increase? That is, should our belief fusion operator not have a neutral opinion. If so, then we choose averaging belief fusion. It means that we are taking the average of the amount of evidence from the two opinions.
\item Otherwise, by having a neutral opinion, we choose weighted belief fusion. It means that we are taking average of the two opinions weighted by confidence.
\end{itemize}

The operator that we will choose has to be $n$-ary, because it allows us multiple interactions in a single iteration. The book \cite{DBLP:books/sp/Josang16} shows only binary fusion operators. The paper \cite{josang2018categories} defines $n$-ary belief fusion operators.


\section{Next steps}

We need to choose with belief fusion operator suits better for our model. It is possible that the operator for fusing all trust-discounted opinions and the operator for fusing the new opinion with the old opinion are different. Simulations with different operators could hep the decision. The repository on GitHub at \cite{oliveira2021subjective} implements some of the operators and it will be used for simulations.


\printbibliography

\end{document}