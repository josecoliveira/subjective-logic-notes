\documentclass[11pt, envcountsect]{beamer}
\usepackage[utf8]{inputenc}
\usepackage[T1]{fontenc}
\usepackage{lmodern}
\usepackage[english]{babel}
\usepackage{amsmath}
\usepackage{amsfonts}
\usepackage{amssymb}
\usepackage{graphicx}
\usetheme{default}

\newenvironment<>{proposition}[1][\undefined]{%
\begin{actionenv}#2%
\ifx#1\undefined%
   \def\insertblocktitle{Theorem}%
\else%
   \def\insertblocktitle{Theorem ({\em#1})}%
\fi%
\par%
\usebeamertemplate{block begin}\em}
{\par\usebeamertemplate{block end}\end{actionenv}}

\begin{document}

\author{José C. Oliveira}
\title{3.5 Multinomial Opinions}
%\subtitle{}
%\logo{}
%\institute{}
%\date{}
%\subject{}
%\setbeamercovered{transparent}
%\setbeamertemplate{navigation symbols}{}
\begin{frame}[plain]
	\maketitle
\end{frame}

\section*{Outline}
\begin{frame}{Summary}
    \tableofcontents
\end{frame}

\section{3.5.1 The Multinomial Option Representation}

\begin{frame}[allowframebreaks]{3.5.1 The Multinomial Opinion Representation}
    \begin{block}{Definition 3.4 (Multinomial Opinion)}
        Let $\mathbb{X}$ be a domain larger than binary, i.e. so that $k = |X| > 2$. Let $X$ be a random variable in $\mathbb{X}$. A multinomial opinion over the random variable $X$ is the ordered triplet $\omega_X = (\mathbf{b}_X, u_X , \mathbf{a}_X)$ where
        \begin{itemize}
            \item $\mathbf{b}_X$ is a belief mass distribution over $X$,
            \item $u_X$ is the uncertainty mass which represents the vacuity of evidence,
            \item $\mathbf{a}_X$ is a base rate distribution over $\mathbb{X}$,
        \end{itemize}
        and the multinomial additivity requirement of Eq.(2.6) is satisfied.
    \end{block} ~\\

    A multinomial opinion has $(2k - 1)$ degrees of freedom.

    \break

    The projected probability distribution of multinomial opinions is defined by:
    \begin{equation}\label{eq:multinomial_projected_probability}\tag{3.12}
       	\mathbf{P}_X(x) = \mathbf{b}_X(x) + \mathbf{a}_X(x) u_X,\ \forall x \in \mathbb{X}\text{.}
    \end{equation}

    The variance of multinomial opinions is expressed as
    \begin{equation}\tag{3.13}
       	\mathrm{Var}_X = \dfrac{\mathbf{P}_X(x)(1 - \mathbf{P}_X(x)u_X)}{W + u_X},
    \end{equation}
    where $W$ denotes non-informative prior weight, which must be set to $W = 2$.
  	\end{frame}

\section{3.5.2 The Dirichlet Multinomial Model}

\begin{frame}{3.5.2 The Dirichlet Multinomial Model}

\end{frame}

\section{3.5.3 Visualising Dirichlet Probability Density Functions}
\section{3.5.4 Coarsening Example: From Ternary to Binary}
\section{3.5.5 Mapping Between Multinomial Opinion and Dirichlet PDF}
\section{3.5.6 Uncertainty-Maximisation}

\end{document}

